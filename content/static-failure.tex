\sectiontitle{Static Failure in Ductile Materials}
\header{Maximum Normal Stress (Coulomb) Theory}
\vspace{-3pt}
\begin{equation*}
    |\sigma|_{max} = \frac{Sy}{n}
\end{equation*}
\header{Maximum Shearing Stress (Tresca) Theory}\vspace{-12pt}
\begin{align*}
    \tau_{max} = \frac{S_{sy}}{n} \text{  where  } S_{sy}=\frac{S_y}{2}
\end{align*}
For the 2D case $\sigma_1\geq\sigma_2$ and $\sigma_3=0$
        \[2(\tau_{max})= \begin{cases} 
              \sigma_1 & \text{when }\sigma_1 \geq \sigma_2 \geq 0\\
                \sigma_2 - \sigma_1 & \text{when } \sigma_1 \geq 0 \geq \sigma_2\\
                \sigma_1 & \text{when } 0\geq \sigma_1 \geq \sigma_2
           \end{cases}
        \]\\
Special 2D Case (shortcut w/ plane stress):
\begin{equation*}
    \sqrt{\sigma^2+4\tau^2}=\frac{S_y}{n}
\end{equation*}
\header{Von Mises}

Most widely used theory for ductile materials
    $S_{sy} = \frac{S_y}{\sqrt{3}}$
    \begin{enumerate}
        \item 3D Case
        \begin{align*}
            \sqrt{\frac{1}{2}(\sigma_1-\sigma_2)^2+(\sigma_1-\sigma_3)^2+(\sigma_2-\sigma_3)^2}= \frac{S_y}{n}
        \end{align*}
        \item 2D Case ($\sigma_3 = 0$)
        \vspace{-10pt}
        \begin{align*}
            \sqrt{\sigma_1^2-\sigma_1\sigma_2+\sigma_2^2}= \frac{S_y}{n}
        \end{align*}
        Special 2D Case
        \vspace{-10pt}
        \begin{align*}
            \sqrt{\sigma^2+3\tau^2}=\frac{S_y}{n}
        \end{align*}
    \end{enumerate}



%------------------------------------------------
\sectiontitle{Static Failure in Brittle Materials}


\header{Maximum Normal Stress Theory for Brittle Materials}
\vspace{-10pt}
\begin{align*}
    \sigma_1 = \frac{S_{ut}}{n}\ \text{or}\ \sigma_1=-\frac{S_{uc}}{n}\  \text{or} \ \sigma_2=\frac{S_{ut}}{n}\ \text{or} \ \sigma_2=-\frac{S_{uc}}{n}
\end{align*}
\header{Coulomb-Mohr Yield Criterion}
Recall the eqn of a line with y-int $b$ and x-int $a$: $\frac{x}{a}+\frac{y}{b} = 1$
\begin{align*}
    \sigma_A&=\frac{S_{ut}}{n}, &\sigma_A\geq \sigma_B \geq 0\\
    \frac{\sigma_A}{S_{ut}}-\frac{\sigma_A}{S_{uc}}&=\frac{1}{n}, &\sigma_A\geq  0\geq \sigma_B\\
    \sigma_B&=-\frac{S_{uc}}{n}, &0\geq \sigma_A\geq \sigma_B 
\end{align*}
\header{Modified Mohr Yield Criterion}
\vspace{-12pt}
\begin{align*}
    \sigma_A&=\frac{S_{ut}}{n} &\sigma_A \geq \sigma_B \geq 0 \\
     &&\sigma_A \geq 0 \geq \sigma_B\text{  and  } \left|\frac{\sigma_B}{\sigma_A}\right|\leq 1\\
    \frac{(S_{uc}-S_{ut})\sigma_A}{S_{uc}S_{ut}}-\frac{\sigma_B}{S_{uc}}&=\frac{1}{n} &\sigma_A \geq 0 \geq \sigma_B\text{  and  }\left|\frac{\sigma_B}{\sigma_A}\right|> 1\\
    \sigma_B &=-\frac{S_{uc}}{n} &0\geq \sigma_A \geq \sigma_B
    \end{align*}
\header{Stress Concentration Factor}
Multiply stresses by $K_t$ when dealing with a brittle material